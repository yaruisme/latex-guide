\documentclass[UTF8]{ctexbook}
\usepackage{a4}

\ctexset{
    section={
        format+=\zihao{-4} \heiti \raggedright,
        name = {,、},
        number = \chinese{section},
        beforeskip = 1.0ex plus 0.2ex minus .2ex,
        afterskip = 1.0ex plus 0.2ex minus .2ex,
        aftername = \hspace{0pt} 
    },subsection={
        format+=\zihao{5} \heiti \raggedright,
        name = {,、},
        number = \arabic{subsection},
        beforeskip = 1.0ex plus 0.2ex minus .2ex,
        afterskip = 1.0ex plus 0.2ex minus .2ex,
        aftername = \hspace{0pt} 
    },
}
\title{ Latex学习}
\author{JiangHao}
\date{\today}
\begin{document}
    \chapter{绪论}
    \section{引言}
    近年来,随着科技飞速发展。人们生活发生了天翻地覆的变化。对此,中国政法大学知识产权研究中心特约研究员李俊慧向观察者网表示:“所谓经营地法律,理论上是在美国经营的主体遵守美国的法律,在国内经营的主体遵守国内的法律。”

    根据禁令要求,任何非美国的芯片制造企业,必须先向美国政府提出申请并获得许可,才可以使用美国的技术和工具给华为供货。

换句话说,美国政府要求包括台积电、中芯国际等在内的芯片制造商不得在未经许可的情况下,采用美国公司的设备生产华为所用的芯片等部件。

因此,未获得美国政府许可的台积电,自5月15日起不能再处理任何来自华为以及华为旗下的海思半导体公司的新订单,并且必须在9月14日之前将原有的订单完成。

事实上,8月7日,中芯国际联合CEO梁孟松曾就上述问题表态,该公司绝对遵守国际规章,有很多其他客户也准备进入其有限的产能里面,若不能继续支持华为海思,影响应该可以控制。

同一天,在中国信息化百人会2020年峰会上,华为消费者业务CEO余承东坦言,由于美国制裁,华为麒麟高端芯片在9月15日之后无法制造,将成为绝唱。
    \section{实验方法}
    \section{实验结果}
    \subsection{数据}
    \subsection{图表}
    \subsubsection{实验条件} %subsubsection 在book下无效
    \subsection{结果分析}
    \section{结论}
    \section{致谢}
\end{document}