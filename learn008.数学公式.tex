%导言区
\documentclass{ctexart}
%罗马数字
\usepackage{amssymb}
\makeatletter
\newcommand{\rmnum}[1]{\romannumeral #1}
\newcommand{\Rmnum}[1]{\expandafter\@slowromancap\romannumeral #1@}
\makeatother

%希腊字母
\usepackage{unicode-math}
\usepackage{amsmath}
%大写空心粗体字母
\usepackage{amsfonts}
\begin{document}
    \section{简介}
    \LaTeX 将排版内容分为文本模式和数学模式

        \section{行内公式}
        \subsection{美元符号}
        交换律是 $a+b=b+a$,如$1+2=2+1=3$
        \subsection{小括号}
        交换律是 \(a+b=b+a\),如\(1+2=2+1=3\)
        \subsection{math环境}
        交换律是 \begin{math}a+b=b+a\end{math},如\begin{math}1+2=2+1=3\end{math}

        \section{上下标}

        \subsection{上标}
        $3x^{20} - x + 2 = 0$
        
        公式作为上标
        $3x^{xy^{20}-x+2} - x + 2 = 0$

        \subsection{下标}
        $a_0,a_1,a_2,a_3, ... a_{99},a_{100}$

        \section{罗马数字}
        \rmnum{1}
        
        \rmnum{2} 
        
        \rmnum{3}
        
        \rmnum{4}

        \rmnum{5}

        \Rmnum{1} \Rmnum{2} \Rmnum{3} \Rmnum{4} \Rmnum{5} \Rmnum{6}
        \section{希腊字母}
        \begin{table}[h]
            \centering
            \begin{tabular}{|c|c|c|c|}
                \hline
                小写形式	& 代码	& 大写形式	& 代码\\
                \hline
                $\alpha$ &\textbackslash alpha& $\Alpha$ &  \textbackslash Alpha\\
                \hline
                $\beta$ & \textbackslash beta & $\Beta$ & \textbackslash Beta\\
                \hline
                $\gamma$ & \textbackslash gamma & $\Gamma$ &  \textbackslash Gamma\\
                \hline
                $\delta$ & \textbackslash delta & $\Delta$ & \textbackslash Delta\\
                \hline
                $\epsilon$ & \textbackslash epsilon & $\Epsilon$ & \textbackslash Epsilon\\
                \hline
                $\zeta$ & \textbackslash zeta & $\Zeta$ & \textbackslash Zeta\\
                \hline
                $\eta$ & \textbackslash eta & $\Eta$ & \textbackslash Eta\\
                \hline
                $\theta$ & \textbackslash theta & $\Theta$ & \textbackslash Theta\\
                \hline
                $\iota$ & \textbackslash iota & $\Iota$ & \textbackslash Iota\\
                \hline
                $\kappa$ & \textbackslash kappa & $\Kappa$ & \textbackslash Kappa\\
                \hline
                $\lambda$ & \textbackslash lambda & $\Lambda$ & \textbackslash Lambda\\
                \hline
                $\mu$ & \textbackslash mu & $\Mu$ & \textbackslash Mu\\
                \hline
                $\nu$ & \textbackslash nu & $\Nu$ & \textbackslash Nu\\
                \hline
                $\xi$ & \textbackslash xi & $\Xi$ & \textbackslash Xi\\
                \hline
                $\omicron$ & \textbackslash omicron & $\Omicron$ & \textbackslash Omicron\\
                \hline
                $\pi$ & \textbackslash pi & $\Pi$ & \textbackslash Pi\\
                \hline
                $\rho$ & \textbackslash rho & $\Rho$ & \textbackslash Rho\\
                \hline
                $\sigma$ & \textbackslash sigma & $\Sigma$ & \textbackslash Sigma\\
                \hline
                $\tau$ & \textbackslash tau & $\Tau$ & \textbackslash Tau\\
                \hline
                $\upsilon$ & \textbackslash tau & $\Upsilon$ & \textbackslash Upsilon\\
                \hline
                $\phi \varphi$ & \textbackslash phi \textbackslash varphi & $\Phi \varPhi$ & \textbackslash Phi \textbackslash varPhi\\
                \hline
                $\chi $ & \textbackslash chi \textbackslash Chi & $\Chi$ & \textbackslash Chi \\
                \hline
                $\psi $ & \textbackslash psi \textbackslash Psi & $\Psi$ & \textbackslash Psi \\
                
                \hline
            \end{tabular}
        \end{table}

        $\alpha$ $\beta$ $\gamma$ $\epsilon$
    

        $\pi$ $\omega$ $\delta$
        
        $\Gamma$ $\Delta$ $\Theta$ $\Pi$ $\Omega$
        
        \section{数学函数}
        $\log$
        $\sin$
        $\cos$
        $\arccos$
        $\arcsin$

        $\sqrt[3]{4}$
        $\sqrt{4+\sqrt{2}}$
        
        \section{分式}
        
        大约是原体积的 $3/4$
        大约是原体积的 $\frac{3}{4}$
        
        $\frac{\sqrt{5}}{\arccos{45}}$
        \section{行间公式}
            \subsection{美元符号}
            交换律是
            $$a+b=b+a$$
            如$$1+2=2+1=3$$
            \subsection{中括号}
            交换律是
            \[a+b=b+a\]
            如\[1+2=2+1=3\]
            \subsection{dislpaymath环境}
            交换律是
            \begin{displaymath}
            a+b=b+a
            \end{displaymath}
            如
            \begin{displaymath}
                1+2=2+1=3
            \end{displaymath}
            \subsection{自动编号公式equation环境}
                \begin{equation}
                    a+b = b+a
                \end{equation}
            \subsection{不编号公式equation*环境}
                \begin{equation*}
                    a+b = b+a    
                \end{equation*}    
\end{document}